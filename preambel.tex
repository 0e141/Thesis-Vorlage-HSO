% Document type and used packages
\documentclass[open=right, % Sorgt für Umbruch bei Chapter (any erzeugt keine Leerseiten) -> Kapitel darf nur auf der rechten Seite beginnen
    paper=A4,               % DIN-A4-Papier
    a4paper,                % DIN-A4-Papier
    12pt,                   % Schriftgöße
    headings=small,         % Kleine Überschriften
    headsepline=true,       % Trennlinie am Kopf der Seite
    footsepline=false,      % Keine Trennlinie am Fuß der Seite
    bibliography=totoc,     % Literaturverzeichnis in das Inhaltsverzeichnis aufnehmen
    twoside=off,            % Für doppelseitigen Druck auf on stellen, off für einseitig
    DIV=7,                  % Verhältnis der Ränder zum bedruckten Bereich
    chapterprefix=false,     % Kapitel x vor dem Kapitelnamen
    numbers=noendperiod,    % kein Punkt am Ende der Nummerierung
    %cleardoublepage=plain
    ]{scrbook}
    
\raggedbottom

% Pakete einbinden, die benötigt werden
\usepackage{scrlayer-scrpage}
\usepackage[utf8]{inputenc}       % Dateien in UTF-8 benutzen
\usepackage[T1]{fontenc}          % Zeichenkodierung
\usepackage{graphicx}             % Bilder einbinden
\usepackage[ngerman]{babel}       % Deutsch und Englisch unterstützen
%\usepackage[english]{babel}       % Deutsch und Englisch unterstützen
\usepackage{xcolor}               % Color support https://www.overleaf.com/project/60a27f64c55a8e1d7a1689cb
\usepackage{amsmath}              % Matheamtische Formeln
\usepackage{amsfonts}             % Mathematische Zeichensätze
\usepackage{amssymb}              % Mathematische Symbole
\usepackage{float}                % Fließende Objekte (Tabellen, Grafiken etc.)
\usepackage{hhline, booktabs}     % Korrekter Tabellensatz
\usepackage[printonlyused, withpage, footnote]{acronym}  % Abkürzungsverzeichnis [nur verwendete Abkürzugen]
\usepackage{makeidx}              % Sachregister
\usepackage{listings}             % Source Code listings
\usepackage{listingsutf8}         % Listings in UTF8
\usepackage[hang,font={sf,footnotesize},labelfont={footnotesize,bf}]{caption} % Beschriftungen
\usepackage[scaled]{helvet}       % Schrift Helvetia laden
\usepackage[absolute]{textpos}	  % Absolute Textpositionen (für Deckblatt)
\usepackage{calc}                 % Berechnung von Positionen
\usepackage{blindtext}            % Blindtexte
%\usepackage[bottom=40mm,left=35mm,right=35mm,top=30mm]{geometry} % Ränder ändern
\usepackage[left=30mm,right=25mm,top=25mm,bottom=25mm]{geometry} % Ränder ändern
\usepackage{setspace}             % Abstände korrigieren
\usepackage{ifthen}               % Logische Bedingungen mit ifthenelse
\usepackage{scrhack}              % Get rid of tocbasic warnings
\usepackage[pagebackref=false,german]{hyperref}  % Hyperlinks
%\usepackage[pagebackref=false,english]{hyperref}  % Hyperlinks
\usepackage[all]{hypcap}          % Korrekte Verlinkung von Floats
\usepackage[autostyle=true,german=quotes]{csquotes}   % Zitate
\usepackage[backend=biber,
  %isbn=true,                     % ISBN nicht anzeigen, gleiches geht mit nahezu allen anderen Feldern
  %sortlocale=de_DE,              % Sortierung der Einträge für Deutsch
  %sortlocale=en_US,              % Sortierung der Einträge für Englisch
  autocite=inline,                % regelt Aussehen für \autocite (inline=\parancite)
  hyperref=true,                  % Hyperlinks für Zitate
  bibstyle=ieee,                  % Quellenverzeichnis im IEEE Style [1]
  citestyle=numeric               % Zitate als Zahlen [1] und [1,2]
]{biblatex}                       % Literaturverwaltung mit BibLaTeX
\usepackage{rotating}             % Seiten drehen
\usepackage{harveyballs}          % Harveyballs
\usepackage{tcolorbox}
\usepackage[export]{adjustbox}
\usepackage{subcaption}
\usepackage{color}
\usepackage{colortbl}
\usepackage{wrapfig}
\usepackage{todonotes}
\usepackage{tabularx}
%\newcolumntype{b}{>{\hsize=1.2\hsize}X}
%\newcolumntype{m}{>{\hsize=.5\hsize}X}
%\newcolumntype{s}{>{\hsize=.3\hsize}X}
\usepackage{tikz}
\usepackage[toc, acronym]{glossaries} %Glossaries und Abkürzungen
\usepackage{etoolbox} %Für Abstände innerhalb des Inhaltsverzeichnisses
%\usepackage{lscape}
\usepackage{pdflscape}
\usepackage{diagbox}
\usepackage{silence}
\usepackage{bookmark}
\usepackage{array}


\setlength{\bibitemsep}{1em}     % Abstand zwischen den Literaturangaben
\setlength{\bibhang}{2em}        % Einzug nach jeweils erster Zeile

% Trennung von URLs im Literaturverzeichnis (große Werte [> 10000] verhindern die Trennung)
\defcounter{biburlnumpenalty}{10} % Strafe für Trennung in URL nach Zahl
\defcounter{biburlucpenalty}{500}  % Strafe für Trennung in URL nach Großbuchstaben
\defcounter{biburllcpenalty}{500}  % Strafe für Trennung in URL nach Kleinbuchstaben

% Farben definieren
\definecolor{linkblue}{RGB}{0, 0, 100}
\definecolor{linkblack}{RGB}{0, 0, 0}
%\definecolor{linkblack}{RGB}{255, 0, 0}
\definecolor{comment}{RGB}{63, 127, 95}
\definecolor{darkgreen}{RGB}{14, 144, 102}
\definecolor{darkblue}{RGB}{0,0,168}
\definecolor{darkred}{RGB}{128,0,0}
\definecolor{javadoccomment}{RGB}{0,0,240}
\definecolor{Gray}{RGB}{242,242,242}

% Einstellungen für das Hyperlink-Paket
\hypersetup{
    colorlinks=true,      % Farbige links verwenden
%    allcolors=linkblue,
    linktoc=all,          % Links im Inhaltsverzeichnis
    linkcolor=linkblack,  % Querverweise
    citecolor=linkblack,  % Literaturangaben
	filecolor=linkblack,  % Dateilinks
	urlcolor=linkblack    % URLs
}

% Einstellungen für Quelltexte
\definecolor{backcolour}{rgb}{0.95,0.95,0.92}
\definecolor{codegray}{rgb}{0.5,0.5,0.5}
\lstset{
      xleftmargin=0.1cm,
      basicstyle=\footnotesize\ttfamily,
      keywordstyle=\color{darkgreen},
      identifierstyle=\color{darkblue},
      commentstyle=\color{comment},
      stringstyle=\color{darkred},
      tabsize=2,
      lineskip={2pt},
      columns=flexible,
      inputencoding=utf8,
      captionpos=b,
      backgroundcolor=\color{backcolour},   
      breakautoindent=true,
	   breakindent=2em,
	   breaklines=true,
	   prebreak=,
	   postbreak=,
      numbers=left,                    
      numbersep=5pt,  
      numberstyle=\tiny\color{codegray},  
      showspaces=false,      % Keine Leerzeichensymbole
      showtabs=false,        % Keine Tabsymbole
      showstringspaces=false,% Leerzeichen in Strings
      morecomment=[s][\color{javadoccomment}]{/**}{*/},
      literate={Ö}{{\"O}}1 {Ä}{{\"A}}1 {Ü}{{\"U}}1 {ß}{{\ss}}2 {ü}{{\"u}}1 {ä}{{\"a}}1 {ö}{{\"o}}1
}


\urlstyle{same}

% Einstellungen für Überschriften
\renewcommand*{\chapterformat}{%
  \Large~\thechapter ~   		    % Große Schrift ohne Punkt am Ende
  \vspace{0.3cm}               	% Abstand zum Titel des Kapitels
}

% Abstände für die Überschriften setzen
\renewcommand{\chapterheadstartvskip}{\vspace*{2.6cm}}
\renewcommand{\chapterheadendvskip}{\vspace*{1.5cm}}

\RedeclareSectionCommand[
  beforeskip=-1.8\baselineskip,
  afterskip=0.25\baselineskip]{section}

\RedeclareSectionCommand[
  beforeskip=-1.8\baselineskip,
  afterskip=0.15\baselineskip]{subsection}

\RedeclareSectionCommand[
  beforeskip=-1.8\baselineskip,
  afterskip=0.15\baselineskip]{subsubsection}


% In der Kopfzeile nur die kurze Kapitelbezeichnung (ohne Kapitel davor)
\renewcommand*\chaptermarkformat{\thechapter\autodot\enskip}
\automark[chapter]{chapter}


% Einstellungen für Schriftarten
\setkomafont{pagehead}{\normalfont\sffamily}
\setkomafont{pagenumber}{\normalfont\sffamily}
\setkomafont{paragraph}{\sffamily\bfseries\small}
\setkomafont{subsubsection}{\sffamily\itshape\bfseries\small}
\addtokomafont{footnote}{\footnotesize}
\setkomafont{chapter}{\LARGE\selectfont\bfseries}

% Wichtige Abstände
\setlength{\parskip}{0.2cm}  % 2mm Abstand zwischen zwei Absätzen
\setlength{\parindent}{0mm}  % Absätze nicht einziehen
\clubpenalty = 10000         % Keine "Schusterjungen"
\widowpenalty = 10000        % Keine "Hurenkinder"  
\displaywidowpenalty = 10000 % Keine "Hurenkinder"
\renewcommand{\footnotesize}{\fontsize{9}{10}\selectfont} % Größe der Fußnoten
\setlength{\footnotesep}{8pt} % Abstand zwischen den Fußnoten

% Abstände für die Überschriften innerhalb des Inhaltsverzeichnisses
\makeatletter
\pretocmd{\chapter}{\addtocontents{toc}{\protect\addvspace{30\p@}}}{}{}
\pretocmd{\section}{\addtocontents{toc}{\protect\addvspace{10\p@}}}{}{}

% Spacing für Roman Page numbers
\renewcommand*{\@pnumwidth}{3.5em}
\renewcommand*{\@tocrmarg}{4.5em}
\makeatother

% Index erzeugen
\makeindex

% Einfacher Font-Wechsel über dieses Makro
\newcommand{\changefont}[3]{
\fontfamily{#1} \fontseries{#2} \fontshape{#3} \selectfont}

% Eigenes Makro für Bilder
\newcommand{\bild}[4]{
\begin{figure}[H]
  \centering
  \includegraphics[width=#2]{#1}
  \caption[#3]{#4}
  \label{fig:#1}
\end{figure}}

% Wo liegt Sourcecode?
\newcommand{\srcloc}{src/}

% Wo sind die Bilder?
\graphicspath{{bilder/}}

% Makros für typographisch korrekte Abkürzungen
\newcommand{\zb}[0]{z.\,B.\ }
\newcommand{\dahe}[0]{d.\,h.\ }
\newcommand{\ua}[0]{u.\,a.\ }

\newcommand{\proof}[0]{\todo[color=red!50]{Quelle!?}}
\newcommand{\contentFill}[0]{\todo[inline, color=green!40]{Mit Inhalt füllen!}}
\newcommand{\content}[1]{\todo[inline, color=green!40]{#1}}
\newcommand{\optional}[1]{\todo[inline, color=blue!40]{Optional: #1}}

% Flags für Veröffentlichung und Sperrvermerk
\newboolean{hsmapublizieren}
\newboolean{hsmasperrvermerk}

%Glossaries und Abkürzungen erzeugen
\makenoidxglossaries

%Turn off Warnings of Todo Notes
\WarningFilter{latex}{Marginpar on page}