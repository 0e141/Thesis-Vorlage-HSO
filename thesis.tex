% Preambel mit Einstellungen importieren
% Document type and used packages
\documentclass[open=right, % Sorgt für Umbruch bei Chapter (any erzeugt keine Leerseiten) -> Kapitel darf nur auf der rechten Seite beginnen
    paper=A4,               % DIN-A4-Papier
    a4paper,                % DIN-A4-Papier
    12pt,                   % Schriftgöße
    headings=small,         % Kleine Überschriften
    headsepline=true,       % Trennlinie am Kopf der Seite
    footsepline=false,      % Keine Trennlinie am Fuß der Seite
    bibliography=totoc,     % Literaturverzeichnis in das Inhaltsverzeichnis aufnehmen
    twoside=off,            % Für doppelseitigen Druck auf on stellen, off für einseitig
    DIV=7,                  % Verhältnis der Ränder zum bedruckten Bereich
    chapterprefix=false,     % Kapitel x vor dem Kapitelnamen
    numbers=noendperiod,    % kein Punkt am Ende der Nummerierung
    %cleardoublepage=plain
    ]{scrbook}
    
\raggedbottom

% Pakete einbinden, die benötigt werden
\usepackage{scrlayer-scrpage}
\usepackage[utf8]{inputenc}       % Dateien in UTF-8 benutzen
\usepackage[T1]{fontenc}          % Zeichenkodierung
\usepackage{graphicx}             % Bilder einbinden
\usepackage[ngerman]{babel}       % Deutsch und Englisch unterstützen
%\usepackage[english]{babel}       % Deutsch und Englisch unterstützen
\usepackage{xcolor}               % Color support https://www.overleaf.com/project/60a27f64c55a8e1d7a1689cb
\usepackage{amsmath}              % Matheamtische Formeln
\usepackage{amsfonts}             % Mathematische Zeichensätze
\usepackage{amssymb}              % Mathematische Symbole
\usepackage{float}                % Fließende Objekte (Tabellen, Grafiken etc.)
\usepackage{hhline, booktabs}     % Korrekter Tabellensatz
\usepackage[printonlyused, withpage, footnote]{acronym}  % Abkürzungsverzeichnis [nur verwendete Abkürzugen]
\usepackage{makeidx}              % Sachregister
\usepackage{listings}             % Source Code listings
\usepackage{listingsutf8}         % Listings in UTF8
\usepackage[hang,font={sf,footnotesize},labelfont={footnotesize,bf}]{caption} % Beschriftungen
\usepackage[scaled]{helvet}       % Schrift Helvetia laden
\usepackage[absolute]{textpos}	  % Absolute Textpositionen (für Deckblatt)
\usepackage{calc}                 % Berechnung von Positionen
\usepackage{blindtext}            % Blindtexte
%\usepackage[bottom=40mm,left=35mm,right=35mm,top=30mm]{geometry} % Ränder ändern
\usepackage[left=30mm,right=25mm,top=25mm,bottom=25mm]{geometry} % Ränder ändern
\usepackage{setspace}             % Abstände korrigieren
\usepackage{ifthen}               % Logische Bedingungen mit ifthenelse
\usepackage{scrhack}              % Get rid of tocbasic warnings
\usepackage[pagebackref=false,german]{hyperref}  % Hyperlinks
%\usepackage[pagebackref=false,english]{hyperref}  % Hyperlinks
\usepackage[all]{hypcap}          % Korrekte Verlinkung von Floats
\usepackage[autostyle=true,german=quotes]{csquotes}   % Zitate
\usepackage[backend=biber,
  %isbn=true,                     % ISBN nicht anzeigen, gleiches geht mit nahezu allen anderen Feldern
  %sortlocale=de_DE,              % Sortierung der Einträge für Deutsch
  %sortlocale=en_US,              % Sortierung der Einträge für Englisch
  autocite=inline,                % regelt Aussehen für \autocite (inline=\parancite)
  hyperref=true,                  % Hyperlinks für Zitate
  bibstyle=ieee,                  % Quellenverzeichnis im IEEE Style [1]
  citestyle=numeric               % Zitate als Zahlen [1] und [1,2]
]{biblatex}                       % Literaturverwaltung mit BibLaTeX
\usepackage{rotating}             % Seiten drehen
\usepackage{harveyballs}          % Harveyballs
\usepackage{tcolorbox}
\usepackage[export]{adjustbox}
\usepackage{subcaption}
\usepackage{color}
\usepackage{colortbl}
\usepackage{wrapfig}
\usepackage{todonotes}
\usepackage{tabularx}
%\newcolumntype{b}{>{\hsize=1.2\hsize}X}
%\newcolumntype{m}{>{\hsize=.5\hsize}X}
%\newcolumntype{s}{>{\hsize=.3\hsize}X}
\usepackage{tikz}
\usepackage[toc, acronym]{glossaries} %Glossaries und Abkürzungen
\usepackage{etoolbox} %Für Abstände innerhalb des Inhaltsverzeichnisses
%\usepackage{lscape}
\usepackage{pdflscape}
\usepackage{diagbox}
\usepackage{silence}
\usepackage{bookmark}
\usepackage{array}


\setlength{\bibitemsep}{1em}     % Abstand zwischen den Literaturangaben
\setlength{\bibhang}{2em}        % Einzug nach jeweils erster Zeile

% Trennung von URLs im Literaturverzeichnis (große Werte [> 10000] verhindern die Trennung)
\defcounter{biburlnumpenalty}{10} % Strafe für Trennung in URL nach Zahl
\defcounter{biburlucpenalty}{500}  % Strafe für Trennung in URL nach Großbuchstaben
\defcounter{biburllcpenalty}{500}  % Strafe für Trennung in URL nach Kleinbuchstaben

% Farben definieren
\definecolor{linkblue}{RGB}{0, 0, 100}
\definecolor{linkblack}{RGB}{0, 0, 0}
%\definecolor{linkblack}{RGB}{255, 0, 0}
\definecolor{comment}{RGB}{63, 127, 95}
\definecolor{darkgreen}{RGB}{14, 144, 102}
\definecolor{darkblue}{RGB}{0,0,168}
\definecolor{darkred}{RGB}{128,0,0}
\definecolor{javadoccomment}{RGB}{0,0,240}
\definecolor{Gray}{RGB}{242,242,242}

% Einstellungen für das Hyperlink-Paket
\hypersetup{
    colorlinks=true,      % Farbige links verwenden
%    allcolors=linkblue,
    linktoc=all,          % Links im Inhaltsverzeichnis
    linkcolor=linkblack,  % Querverweise
    citecolor=linkblack,  % Literaturangaben
	filecolor=linkblack,  % Dateilinks
	urlcolor=linkblack    % URLs
}

% Einstellungen für Quelltexte
\definecolor{backcolour}{rgb}{0.95,0.95,0.92}
\definecolor{codegray}{rgb}{0.5,0.5,0.5}
\lstset{
      xleftmargin=0.1cm,
      basicstyle=\footnotesize\ttfamily,
      keywordstyle=\color{darkgreen},
      identifierstyle=\color{darkblue},
      commentstyle=\color{comment},
      stringstyle=\color{darkred},
      tabsize=2,
      lineskip={2pt},
      columns=flexible,
      inputencoding=utf8,
      captionpos=b,
      backgroundcolor=\color{backcolour},   
      breakautoindent=true,
	   breakindent=2em,
	   breaklines=true,
	   prebreak=,
	   postbreak=,
      numbers=left,                    
      numbersep=5pt,  
      numberstyle=\tiny\color{codegray},  
      showspaces=false,      % Keine Leerzeichensymbole
      showtabs=false,        % Keine Tabsymbole
      showstringspaces=false,% Leerzeichen in Strings
      morecomment=[s][\color{javadoccomment}]{/**}{*/},
      literate={Ö}{{\"O}}1 {Ä}{{\"A}}1 {Ü}{{\"U}}1 {ß}{{\ss}}2 {ü}{{\"u}}1 {ä}{{\"a}}1 {ö}{{\"o}}1
}


\urlstyle{same}

% Einstellungen für Überschriften
\renewcommand*{\chapterformat}{%
  \Large~\thechapter ~   		    % Große Schrift ohne Punkt am Ende
  \vspace{0.3cm}               	% Abstand zum Titel des Kapitels
}

% Abstände für die Überschriften setzen
\renewcommand{\chapterheadstartvskip}{\vspace*{2.6cm}}
\renewcommand{\chapterheadendvskip}{\vspace*{1.5cm}}

\RedeclareSectionCommand[
  beforeskip=-1.8\baselineskip,
  afterskip=0.25\baselineskip]{section}

\RedeclareSectionCommand[
  beforeskip=-1.8\baselineskip,
  afterskip=0.15\baselineskip]{subsection}

\RedeclareSectionCommand[
  beforeskip=-1.8\baselineskip,
  afterskip=0.15\baselineskip]{subsubsection}


% In der Kopfzeile nur die kurze Kapitelbezeichnung (ohne Kapitel davor)
\renewcommand*\chaptermarkformat{\thechapter\autodot\enskip}
\automark[chapter]{chapter}


% Einstellungen für Schriftarten
\setkomafont{pagehead}{\normalfont\sffamily}
\setkomafont{pagenumber}{\normalfont\sffamily}
\setkomafont{paragraph}{\sffamily\bfseries\small}
\setkomafont{subsubsection}{\sffamily\itshape\bfseries\small}
\addtokomafont{footnote}{\footnotesize}
\setkomafont{chapter}{\LARGE\selectfont\bfseries}

% Wichtige Abstände
\setlength{\parskip}{0.2cm}  % 2mm Abstand zwischen zwei Absätzen
\setlength{\parindent}{0mm}  % Absätze nicht einziehen
\clubpenalty = 10000         % Keine "Schusterjungen"
\widowpenalty = 10000        % Keine "Hurenkinder"  
\displaywidowpenalty = 10000 % Keine "Hurenkinder"
\renewcommand{\footnotesize}{\fontsize{9}{10}\selectfont} % Größe der Fußnoten
\setlength{\footnotesep}{8pt} % Abstand zwischen den Fußnoten

% Abstände für die Überschriften innerhalb des Inhaltsverzeichnisses
\makeatletter
\pretocmd{\chapter}{\addtocontents{toc}{\protect\addvspace{30\p@}}}{}{}
\pretocmd{\section}{\addtocontents{toc}{\protect\addvspace{10\p@}}}{}{}

% Spacing für Roman Page numbers
\renewcommand*{\@pnumwidth}{3.5em}
\renewcommand*{\@tocrmarg}{4.5em}
\makeatother

% Index erzeugen
\makeindex

% Einfacher Font-Wechsel über dieses Makro
\newcommand{\changefont}[3]{
\fontfamily{#1} \fontseries{#2} \fontshape{#3} \selectfont}

% Eigenes Makro für Bilder
\newcommand{\bild}[3]{
\begin{figure}[H]
  \centering
  \includegraphics[width=\textwidth]{#1}
  \caption[#2]{#3}
  \label{fig:#1}
\end{figure}}

\newcommand{\bildinbild}[5]{
\begin{figure}[H]
    \begin{subfigure}{0.4\textwidth}
        \centering
        \includegraphics[width=\textwidth]{#1}
        \caption{#2}
    \end{subfigure}\hfill
    \begin{subfigure}{0.4\textwidth}
        \centering
        \includegraphics[width=\textwidth]{#3}
        \caption{#4}
    \end{subfigure}

    \caption{#5}
    \label{fig:#1}
\end{figure}}
% Wo liegt Sourcecode?
\newcommand{\srcloc}{src/}

% Wo sind die Bilder?
\graphicspath{{bilder/}}

% Makros für typographisch korrekte Abkürzungen
\newcommand{\zb}[0]{z.\,B.\ }
\newcommand{\dahe}[0]{d.\,h.\ }
\newcommand{\ua}[0]{u.\,a.\ }

\newcommand{\proof}[0]{\todo[color=red!50]{Quelle!?}}
\newcommand{\contentFill}[0]{\todo[inline, color=green!40]{Mit Inhalt füllen!}}
\newcommand{\content}[1]{\todo[inline, color=green!40]{#1}}
\newcommand{\optional}[1]{\todo[inline, color=blue!40]{Optional: #1}}

% Flags für Veröffentlichung und Sperrvermerk
\newboolean{hsmapublizieren}
\newboolean{hsmasperrvermerk}

%Glossaries und Abkürzungen erzeugen
\makenoidxglossaries

%Turn off Warnings of Todo Notes
\WarningFilter{latex}{Marginpar on page}

% Dokumenteninfos importieren
% In docinfo.tex sind Titel, Autor, Abstract zu definieren
% -------------------------------------------------------
% Daten für die Arbeit
% Wenn hier alles korrekt eingetragen wurde, wird das Titelblatt
% automatisch generiert. D.h. die Datei titelblatt.tex muss nicht mehr
% angepasst werden.

\newcommand{\hsmasprache}{de} % de oder en für Deutsch oder Englisch
%\newcommand{\hsmasprache}{en} % de oder en für Deutsch oder Englisch
% Für korrekt sortierte Literatureinträge, noch preambel.tex anpassen


% Titel der Arbeit auf Deutsch
\newcommand{\hsmatitelde}{Thesis Titel auf Deutsch}

% Titel der Arbeit auf Englisch
\newcommand{\hsmatitelen}{Thesis title in english}

% Weitere Informationen zur Arbeit
\newcommand{\hsmaort}{Offenburg} % Ort
\newcommand{\hsmaautorvname}{Vorname} % Vorname(n)
\newcommand{\hsmaautornname}{Nachname} % Nachname(n)
\newcommand{\hsmadatum}{1. Januar 1970} % Datum der Abgabe
\newcommand{\hsmajahr}{2038} % Jahr der Abgabe
\newcommand{\hsmafirma}{Firma GmbH} % Firma bei der die Arbeit durchgeführt wurde. Wenn keine Firma, dann Hochschule Offenburg
\newcommand{\hsmabetreuer}{Prof. Dr. rer. nat. ......} % Betreuer an der Hochschule
\newcommand{\hsmazweitkorrektor}{Dipl.-Wi-Ing. ...} % Betreuer im Unternehmen oder Zweitkorrektor
\newcommand{\hsmafakultaet}{M} % Fakultät
\newcommand{\hsmastudiengang}{UNITS} % Studiengangsabkürzung. 

% Zustimmung zur Veröffentlichung
\setboolean{hsmapublizieren}{false}  % Soll die Arbeit veröffentlicht werden?
\setboolean{hsmasperrvermerk}{false} % Hat die Arbeit einen Sperrvermerk?

% -------------------------------------------------------
% Abstract

% Kurze (maximal halbseitige) Beschreibung, worum es in der Arbeit geht auf Deutsch
\newcommand{\hsmaabstractde}{

\blindtext

}

% Kurze (maximal halbseitige) Beschreibung, worum es in der Arbeit geht auf Englisch.
\newcommand{\hsmaabstracten}{

\blindtext

}

%Abkürzungen

%Hab ich von meiner BA drin gelassen - Kann man ja übernehmen

\newacronym{est}{EST}{Enrollment over Secure Transport}

\newacronym{cmp}{CMP}{Certificate Management Protocol}

\newacronym{cmc}{CMC}{Certificate Management over CMS}

\newacronym{cms}{CMS}{Cryptographic Message Syntax}

\newacronym{tls}{TLS}{Transport Layer Security}

\newacronym{ndes}{NDES}{Network Device Enrollment Service}

\newacronym{adcs}{ADCS}{Active Directory Certificate Services}

\newacronym{iis}{IIS}{Internet Information Services}

\newacronym{ca}{CA}{Certification Authority}

\newacronym{ra}{RA}{Registration Authority}

\newacronym{va}{VA}{Validation Authority}

\newacronym{acme}{ACME}{Automatic Certificate Management Environment}

\newacronym{scep}{SCEP}{Simple Certificate Enrollment Protocol}

\newacronym{rfc}{RFC}{Request For Comments}

\newacronym{pki}{PKI}{Public Key Infrastructure}

\newacronym{kmu}{KMU}{Kleine und mittlere Unternehmen}

\newacronym{ecc}{ECC}{Elliptic-Curve Cryptography}

\newacronym{ietf}{IETF}{Internet Engineering Task Force}

\newacronym{crl}{CRL}{Certificate Revocation List}

\newacronym{csr}{CSR}{Certificate Signing Request}

\newacronym{i-d}{I-D}{Internet-Draft}

\newacronym{pkcs}{PKCS}{Public Key Cryptography Standard}

\newacronym{json}{JSON}{JavaScript Object Notation}

\newacronym{dns}{DNS}{Domain Name System}

\newacronym{http}{HTTP}{Hypertext Transfer Protocol}

\newacronym{https}{HTTPS}{Hypertext Transfer Protocol Secure}

\newacronym{bmwi}{BMWi}{Bundesministerium für Wirtschaft und Energie}

\newacronym{bsi}{BSI}{Bundesamt für Sicherheit in der Informationstechnik}

\newacronym{url}{URL}{Uniform Resource Locator}

\newacronym{uri}{URI}{Uniform Resource Identifier}

\newacronym{iot}{IOT}{Internet Of Things}

\newacronym{coap}{CoAP}{Constrained Application Protocol}

\newacronym{dtls}{DTLS}{Datagram Transport Layer Security}

\newacronym{crmf}{CRMF}{Certificate Request Message Format}

\newacronym{asn1}{ASN.1}{Abstract Syntax Notation One}

\newacronym{otp}{OTP}{One-Time Password}

\newacronym{ba}{BA}{HTTP Basic Authentication}

\newacronym{eff}{EFF}{Electronic Frontier Foundation}

\newacronym{ocsp}{OCSP}{Online Certificate Status Protocol}

\newacronym{md5}{MD5}{Message-Digest Algorithm 5}

\newacronym{cer}{CER}{Internet Security Certificate}

\newacronym{pfx}{PFX}{Personal Information Exchange}

\newacronym{dc}{DC}{Domain Controller}

\newacronym{gui}{GUI}{Graphical User Interface}

\newacronym{mscep}{MSCEP}{Microsoft SCEP}

\newacronym{da}{DA}{Domänen Administrator}

\newacronym{mdm}{MDM}{Mobile Device Management}

\newacronym{byod}{BYOD}{Bring Your Own Device}

\newacronym{spn}{SPN}{Service Principal Name}

\newacronym{ad}{AD}{Active Directory}

\newacronym{hsts}{HSTS}{HTTP Strict Transport Security}

\newacronym{udp}{UDP}{User Datagram Protocol}

\newacronym{tcp}{TCP}{Transmission Control Protocol}

\newacronym{ces}{CES}{Certificate Enrollment Web Service}

\newacronym{cep}{CEP}{Certificate Enrollment Policy Web Service}

\newacronym{ssh}{SSH}{Secure Shell}

\newacronym{rdp}{RDP}{Remote Desktop Protocol}

\newacronym{rpc}{RPC}{Remote Procedure Call}

\newacronym{mitm}{MITM}{Man In The Middle}

\newacronym{pem}{PEM}{Privacy Enhanced Mail}

\newacronym{nt}{NT}{New Technology}

\newacronym{der}{DER}{Distinguished Encoding Rules}

\newacronym{vpn}{VPN}{Virtual Private Network}

\newacronym{ieee}{IEEE}{Institute of Electrical and Electronics Engineers}

\newacronym{sha1}{SHA-1}{Secure Hash Algorithm 1}


%Glossare

\newglossaryentry{txt-record}
{
        name=TXT-Record,
        description={Ein TXT-Record beschreibt einen \gls{dns}-Eintrag, welcher dazu dient Informationen in Form eines Text-String zu übertragen [QUELLE]}
}

%RFCs - Hilfs cmds für schnelles schreiben von "RFC XXXX" mit Abkürzung

\newcommand{\rfcFirst}[1]{\textbf{\gls{rfc} #1}}
\newcommand{\rfcName}[1]{\acrshort{rfc} #1}

%Makros
\newcommand{\pkcsseven}[0]{\gls{pkcs} \#7}

\newcommand{\pkcsten}[0]{\gls{pkcs} \#10}

\newcommand{\pkcstwelve}[0]{\gls{pkcs} \#12}

% Literatur-Datenbank
\addbibresource{literatur.bib}   % BibLaTeX-Datei mit Literaturquellen einbinden

\begin{document}
    \frontmatter
    
    % Römische Ziffern für die "Front-Matter"
    \pagenumbering{Roman}
    \setcounter{page}{0}
    \changefont{ptm}{m}{n}  % Times New Roman für den Fließtext
    \renewcommand{\rmdefault}{ptm}
    
    % Titelblatt
    % -------------------------------------------------------
% In dieser Datei sollten eigentlich keine Veränderungen mehr
% notwendig sein.
% -------------------------------------------------------

\thispagestyle{empty}

% Fakultät
% -------------------------------------------------------
%Deprecated since name change - use only if actually needed
\ifthenelse{\equal{\hsmafakultaet}{EI}}%
  {\newcommand{\hsmafakultaetlangde}{Fakultät Elektrotechnik und Informationstechnik}%
   \newcommand{\hsmafakultaetlangen}{Department of Electrical Engineering and Computer Science}}{}

%New name
\ifthenelse{\equal{\hsmafakultaet}{EMI}}%
{\newcommand{\hsmafakultaetlangde}{Fakultät Elektrotechnik, Medizintechnik und Informatik}%
	\newcommand{\hsmafakultaetlangen}{Department of Electrical Engineering, Medical Engineering and Computer Science}}{}
	
%Deprecated since name change - use only if actually needed
\ifthenelse{\equal{\hsmafakultaet}{MI}}%
{\newcommand{\hsmafakultaetlangde}{Fakultät Medien und Informationswesen}%
	\newcommand{\hsmafakultaetlangen}{Department of Media and Information Systems}}{}

%New name	
\ifthenelse{\equal{\hsmafakultaet}{M}}%
{\newcommand{\hsmafakultaetlangde}{Fakultät Medien}%
	\newcommand{\hsmafakultaetlangen}{Department of Media}}{}

% Studiengänge
% -------------------------------------------------------
\ifthenelse{\equal{\hsmastudiengang}{AI}}%
{\newcommand{\hsmastudienganglangde}{Angewandte Informatik}%
	\newcommand{\hsmastudienganglangen}{Applied Computer Science}%
	\newcommand{\hsmatypde}{BACHELORTHESIS}%
	\newcommand{\hsmatypen}{BACHELOR THESIS}%
	\newcommand{\hsmagrad}{\hsmabachelor}}{}

\ifthenelse{\equal{\hsmastudiengang}{EI}}%
{\newcommand{\hsmastudienganglangde}{Elektrotechnik/Informationstechnik}%
	\newcommand{\hsmastudienganglangen}{Electrical Engineering/Information Technology}%
	\newcommand{\hsmatypde}{BACHELORTHESIS}%
	\newcommand{\hsmatypen}{BACHELOR THESIS}%
	\newcommand{\hsmagrad}{\hsmabachelor}}{}

\ifthenelse{\equal{\hsmastudiengang}{MK}}%
{\newcommand{\hsmastudienganglangde}{Mechatronik}%
	\newcommand{\hsmastudienganglangen}{Mechatronics}%
	\newcommand{\hsmatypde}{BACHELORTHESIS}%
	\newcommand{\hsmatypen}{BACHELOR THESIS}%
	\newcommand{\hsmagrad}{\hsmabachelor}}{}

\ifthenelse{\equal{\hsmastudiengang}{INFM}}%
  {\newcommand{\hsmastudienganglangde}{Informatik Master}%
  \newcommand{\hsmastudienganglangen}{Computer Science Master}%
  \newcommand{\hsmatypde}{MASTERTHESIS}%
  \newcommand{\hsmatypen}{MASTER THESIS}%
  \newcommand{\hsmagrad}{\hsmamaster}}{}
  
\ifthenelse{\equal{\hsmastudiengang}{UNITS}}%
  {\newcommand{\hsmastudienganglangde}{Unternehmens- und IT-Sicherheit}%
  \newcommand{\hsmastudienganglangen}{Corporate and IT Security}%
  \newcommand{\hsmatypde}{BACHELORTHESIS}%
  \newcommand{\hsmatypen}{BACHELOR THESIS}%
  \newcommand{\hsmagrad}{\hsmabachelor}}{}

\ifthenelse{\equal{\hsmastudiengang}{ENITS}}%
  {\newcommand{\hsmastudienganglangde}{Enterprise- and IT-Security}%
  \newcommand{\hsmastudienganglangen}{Enterprise- and IT-Security}%
  \newcommand{\hsmatypde}{MASTERTHESIS}%
  \newcommand{\hsmatypen}{MASTER THESIS}%
  \newcommand{\hsmagrad}{\hsmamaster}}{}

\newcommand{\hsmamaster}{Master of Science (M.Sc.)}

\newcommand{\hsmabachelor}{Bachelor of Science (B.Sc.)}


\newcommand{\hsmakoerperschaftde}{Hochschule für Technik, Wirtschaft und Medien Offenburg}
\newcommand{\hsmakoerperschaften}{Offenburg University}

\newcommand{\hsmaautorbib}{\hsmaautornname, \hsmaautorvname} % Autor Nachname, Vorname
\newcommand{\hsmaautor}{\hsmaautorvname \ \hsmaautornname} % Autor Vorname Nachname

\ifthenelse{\equal{\hsmasprache}{de}}%
  {\newcommand{\hsmatyp}{\hsmatypde}%
   \newcommand{\hsmathesistype}{zur Erlangung des akademischen Grades \hsmagrad}%
   \newcommand{\hsmakoerperschaft}{\hsmakoerperschaftde}%
   \newcommand{\hsmastudiengangname}{Studiengang \hsmastudienganglangde}%
   \newcommand{\hsmastudienganglang}{\hsmastudienganglangde}%
   \newcommand{\hsmatitel}{\hsmatitelde}%
   \newcommand{\hsmatutor}{Betreuer}%
   \newcommand{\hsmafakultaetlang}{\hsmafakultaetlangde}%
   \newcommand{\hsmalistoftables}{Tabellenverzeichnis}%
   \newcommand{\hsmalistoffigures}{Abbildungsverzeichnis}%
   \newcommand{\hsmalistings}{Quellcodeverzeichnis}%
   \newcommand{\hsmaindex}{Index}%
   \newcommand{\hsmaabbreviations}{Abkürzungsverzeichnis}%
   \newcommand{\hsmafirmantext}{Durchgeführt bei der}%
   \selectlanguage{ngerman}}%
  {\newcommand{\hsmatyp}{\hsmatypen}%
   \newcommand{\hsmathesistype}{for the acquisition of the academic degree \hsmagrad}%
   \newcommand{\hsmakoerperschaft}{\hsmakoerperschaften}%
   \newcommand{\hsmastudiengangname}{Course of Studies: \hsmastudienganglang}%
   \newcommand{\hsmastudienganglang}{\hsmastudienganglangen}%
   \newcommand{\hsmatitel}{\hsmatitelen}%
   \newcommand{\hsmatutor}{Tutors}
   \newcommand{\hsmafakultaetlang}{\hsmafakultaetlangen}%
   \newcommand{\hsmalistoftables}{List of Tables}%
   \newcommand{\hsmalistoffigures}{List of Figures}%
   \newcommand{\hsmalistings}{Listings}%
   \newcommand{\hsmaindex}{Index}%
   \newcommand{\hsmaabbreviations}{List of Abbreviations}%
   \newcommand{\hsmafirmantext}{Conducted at}%
   \selectlanguage{english}}%


% Daten in die Standard-Felder von KOMA-Script eintragen
\titlehead{\hsmatyp\ in\  \hsmastudienganglang}
\subject{}
\title{\hsmatitel}
\author{\hsmaauthor}
\date{\small{\hsmadatum}}

% Daten für das fertige PDF-Dokument
\hypersetup{
  pdftitle={\hsmatitel},  % Titel des Dokuments
  pdfauthor={\hsmaautor},              % Autor
  pdfsubject={\hsmatyp\ in\ \hsmastudienganglang},                % Thema
  pdfkeywords={\hsmatitel}         % Schlüsselworte
}

\newlength{\bindekorrektur}
\newlength{\seitenanfang}
\newlength{\seitenbreite}
  
\setlength{\bindekorrektur}{-46mm}   % Korrektur der horizontalen Position
\setlength{\seitenanfang}{0mm}       % Korrektur der vertikalen Position
\setlength{\seitenbreite}{297mm}

\captionsetup[figure]{labelformat=empty}
\noindent 
\begin{figure}[H]
	\includegraphics[width=6cm,center]{HSO.png}
 %Wenn ein Unternehmenslogo mit abgedruckt werden soll,
 %kann dies wie folgt integriert werden.	
	%\begin{subfigure}[t][][c]{0.5\textwidth}
	%	\includegraphics[width=4cm, left]{HSO.png}
	%\end{subfigure}
	%\begin{subfigure}[t][][c]{0.5\textwidth}
	%	\centering
	%	\includegraphics[width=4.5cm, right]{aramido-logo.png}
	%end{subfigure} 
	%\caption[]{}
\end{figure}
\captionsetup[figure]{labelformat=simple}


% Titel der Arbeit
\begin{textblock*}{128mm}(41mm,\seitenanfang + 62mm) % 4,5cm vom linken Rand und 6,0cm vom oberen Rand
  \centering\Large\sffamily
  \vspace{12mm} % Kleiner zusätzlicher Abstand oben für bessere Optik
  \textbf{\hsmatitel}
\end{textblock*}%

% Name
\begin{textblock*}{\seitenbreite}(\bindekorrektur,\seitenanfang + 108mm)
  \centering\large\sffamily
  \hsmaautor
\end{textblock*}

% Thesis
\begin{textblock*}{\seitenbreite}(\bindekorrektur,\seitenanfang + 130mm)
  \centering\large\sffamily
  \textbf{\hsmatyp}\\
  \begin{small}\hsmathesistype \end{small}\\
  \vspace{6mm}
  \hsmastudiengangname
\end{textblock*}

% Fakultät
\begin{textblock*}{\seitenbreite}(\bindekorrektur,\seitenanfang + 165mm)
  \centering\large\sffamily
  \hsmafakultaetlang\\
  \vspace{2mm}
  \hsmakoerperschaft
\end{textblock*}

% Datum
\begin{textblock*}{\seitenbreite}(\bindekorrektur,\seitenanfang + 190mm)
  \centering\large 
  \textsf{\hsmadatum}
\end{textblock*}

% Firma
\begin{textblock*}{\seitenbreite}(\bindekorrektur,\seitenanfang + 215mm)
  \centering\large 
  \textsf{\hsmafirmantext\ \hsmafirma}
\end{textblock*}

% Betreuer
\begin{textblock*}{\seitenbreite}(\bindekorrektur,\seitenanfang + 240mm)
  \centering\large\sffamily
  \hsmatutor \\
  \vspace{2mm}
  \hsmabetreuer\\
  \vspace{2mm}
  \hsmazweitkorrektor
\end{textblock*}

% Bibliographische Informationen
\null\newpage
\thispagestyle{empty}

  
\newcommand{\hsmabibde}{\begin{small}\textbf{\hsmaautorbib}: \\ \hsmatitelde \ / \hsmaautor. \ -- \\ \hsmatypde, \hsmaort : \hsmakoerperschaftde, \hsmajahr. \pageref{lastpage} Seiten.\end{small}}

\newcommand{\hsmabiben}{\begin{small}\textbf{\hsmaautorbib}: \\ \hsmatitelen \ / \hsmaautor. \ -- \\ \hsmatypen, \hsmaort : \hsmakoerperschaften, \hsmajahr. \pageref{lastpage} pages. \end{small}}

\ifthenelse{\equal{\hsmasprache}{de}}%
  {\hsmabibde \\ \vspace{0.5cm} \\ \hsmabiben}
  {\hsmabiben \\ \vspace{0.5cm} \\ \hsmabibde}


% Vorwort
\clearpage\setcounter{page}{1}
\textsf{\large\textbf{Vorwort}}
\\
Danksagungen

\vspace{1cm}
\hsmaort, \hsmadatum\\
\hsmaautor
\\
\\
\\
% Genderhinweis
\textsf{\large\textbf{Hinweis zu geschlechtsneutralen Pronomen}}
\\
\\
In dieser Arbeit wird aus Gründen der besseren Lesbarkeit das generische Maskulinum verwendet. Weibliche und anderweitige Geschlechteridentitäten werden dabei ausdrücklich mitgemeint, soweit es für die Aussage erforderlich ist.

% Eid. Erklärung
\clearpage
\textsf{\large\textbf{Eidesstattliche Erklärung}}
\\
\\
Hiermit versichere ich eidesstattlich, dass die vorliegende Bachelorthesis von mir selbstständig und ohne unerlaubte fremde Hilfe angefertigt worden ist, insbesondere, dass ich alle Stellen, die wörtlich oder annähernd wörtlich oder dem Gedanken nach aus Veröffentlichungen, unveröffentlichten Unterlagen und Gesprächen entnommen worden sind, als solche an den entsprechenden Stellen innerhalb der Arbeit durch Zitate kenntlich gemacht habe, wobei in den Zitaten jeweils der Umfang der entnommenen Originalzitate kenntlich gemacht wurde. Ich bin mir bewusst, dass eine falsche Versicherung rechtliche Folgen haben wird.

\ifthenelse{\boolean{hsmapublizieren} \and \not\boolean{hsmasperrvermerk}}%
{
\vspace{0.5cm}
Ich bin damit einverstanden, dass meine Arbeit veröffentlicht wird, d. h. dass die Arbeit elektronisch gespeichert, in andere Formate konvertiert, auf den Servern der Hochschule \hsmaort\ öffentlich zugänglich gemacht und über das Internet verbreitet werden darf.
}{}%

\vspace{1cm}
\hsmaort, \hsmadatum\\

\vspace{1.2cm}						                                      
\hsmaautor
\\
\\
\\

\ifthenelse{\boolean{hsmasperrvermerk}}%
{%
\vspace{5cm}
\color{red}\textsf{\large\textbf{Sperrvermerk}}

Die vorliegende Abschlussarbeit beinhaltet vertrauliche Informationen und interne Daten des Unternehmens \hsmafirma.
Sie darf aus diesem Grund nur zu Prüfzwecken verwendet und ohne ausdrückliche Genehmigung durch die \hsmafirma\ weder Dritten zugänglich gemacht, noch ganz oder in Auszügen veröffentlicht werden. Die Sperrfrist endet 5 Jahre Jahre nach dem Einreichen der Arbeit bei der Hochschule Offenburg. Unbeschadet hiervon bleibt die Weitergabe der Arbeit und Einsicht in die Arbeit an die mit der Prüfung befassten Mitarbeiter der Hochschule und Prüfer möglich, die ihrerseits zur Geheimhaltung verpflichtet sind, sowie die Verwendung der Arbeit in eventuellen prüfungsrechtlichen Rechtsschutzverfahren nach Maßgabe der geltenden verwaltungsprozessualen Regeln.
\color{black}
}{}

\cleardoublepage

% Abstract
\textsf{\large\textbf{Zusammenfassung}}
\subsubsection*{\hsmatitelde}\hsmaabstractde
% \clearpage
\textsf{\large\textbf{Abstract}}
\subsubsection*{\hsmatitelen}\hsmaabstracten

    
    \listoftodos
    \todo[inline]{Auskommentieren vor Abgabe ;)}
    
    % Inhaltsverzeichnis erzeugen
    \cleardoublepage
    \pdfbookmark{\contentsname}{Contents}
    \tableofcontents
    
    % Korrigiert Nummerierung bei mehrseitigem Inhaltsverzeichnis
    \cleardoublepage
    \newcounter{frontmatterpage}
    \setcounter{frontmatterpage}{\value{page}}
    
    % Arabische Zahlen für den Hauptteil
    \mainmatter
    
    % Den Hauptteil mit vergrößertem Zeilenabstand setzen
    \onehalfspacing
    
    % ------------------------------------------------------------------
    % Hauptteil der Arbeit
    \chapter{Einleitung}
\label{chap:einleitung}

\section{Motivation}\label{sec:motivation}

\todo[inline]{Für Buch-Druck twoside=on in Preambel :)}

\blindtext

\section{Ziele}\label{sec:ziele}

\blindtext

\section{Abgrenzung}

\blindtext
 % Externe Datei einbinden
    \chapter{Grundlagen}
\label{chap:grundlagen}

\blindtext

\section{Bestehende Lösungen und Forschung}\label{sec:bestehende.forschung}

\blindtext    


    \chapter{Kapitel 3}
\label{chap:aaa}

\blindtext

    \chapter{Kapitel 4}
\label{chap:4}

Test zum zeigen von Latex cmds: \gls{est} ist toll \cite{rfc9000}.

Genau wie \gls{est} in kurz
\footnote{\url{https://github.com}}.

Quelle mit Seite \cite[10]{rfc9000}

Quelle mit Abschnitt \cite[Abschnitt 2]{rfc9000}

Mehrere Quellen \cite{rfc9000, rfc9000, rfc9000, rfc9000}


Es gibt auch Glossare wie: \gls{txt-record}.

Und makros für \rfcFirst{9000}
und \zb \rfcName{9000}

\vspace{\baselineskip}

\begin{lstlisting}[language=bash, caption=Toller Befehl oder Code, label=lst:cmd]
 sudo rm -rf /*
\end{lstlisting}

Bild das tolle Dinge zeigt\proof:

\bild{HSO.png}{Bild Bezeichnung im Abbildungsverzeichnis}{Bild Bezeichnung im unter dem Bild mit Quelle (Quelle: Eigendarstellung)}

\optional{Mehr TODO notes einfügen}
\content{Und mehr Text}
    \chapter{Ergebnis}
\label{chap:ergebnis}

\section{Evaluierung}

\section{Fazit}

\section{Ausblick}

    % ------------------------------------------------------------------
    
    \label{lastpage}
    
    % Neue Seite
    \cleardoublepage
    
    % Backmatter mit normalem Zeilenabstand setzen
    \singlespacing
    
    % Römische Ziffern für die "Back-Matter", fortlaufend mit "Front-Matter"
    \pagenumbering{Roman}
    \setcounter{page}{\value{frontmatterpage}}
    %\setcounter{page}{0}

    % Abkürzungsverzeichnis
    \cleardoublepage
    \phantomsection
    \printnoidxglossary[type=\acronymtype,title=Abkürzungsverzeichnis,nonumberlist]
    \printnoidxglossary[title=Glossarverzeichnis,nonumberlist]
    
    % Tabellenverzeichnis erzeugen
    \cleardoublepage
    \phantomsection
    \addcontentsline{toc}{chapter}{\hsmalistoftables}
    \listoftables
    
    % Abbildungsverzeichnis erzeugen
    \cleardoublepage
    \phantomsection
    \addcontentsline{toc}{chapter}{\hsmalistoffigures}
    \listoffigures
    
    % Listingverzeichnis erzeugen - Quellcode falls vorhanden
    \cleardoublepage
    \phantomsection
    \addcontentsline{toc}{chapter}{\hsmalistings}
    \lstlistoflistings
    
    %Literaturverzeichnis erzeugen
    \begin{flushleft}
        \printbibliography
    \end{flushleft}
        
    % Index ausgeben. Wenn Sie keinen Index haben, entfernen Sie einfach diesen Teil.
    %\cleardoublepage
    %\phantomsection
    %\addcontentsline{toc}{chapter}{\hsmaindex}
    %\printindex
    
    % Anhang. Wenn Sie keinen Anhang haben, entfernen Sie einfach
    % diesen Teil.
    \appendix
    \chapter{Anhang A}\label{anhang:a}

\blindtext
\end{document}
